%%%%%%%%%%%%%%%%%%%%%%%%%%%%%%%%%%%%%%%%%%%%%%%%%%%%%%%%%%%%%%%%%%%%%%
% Problem statement
\begin{statement}[
  problempoints=100,
  timelimit=1 sekunda,
  memorylimit=512 MiB,
]{Automatizacija}

Životni je cilj uspješne poduzetnice Elene Mošus zamijeniti svu ljudsku radnu snagu 
umjetnom inteligencijom. 
Kako bi ubrzala taj proces, zaključila je da je potrebno uključiti se u hrvatsko zakonodavstvo. 
Njezin naum došao je do  
predsjednika države, koji ju je postavio na čelo novonastalog 
Ministarstva automatizacije logičkih načela i analitičkog razmišljanja 
(skraćeno MALNAR). 
Njihov je prvi zadatak automatizirati sljedeću igru. 

Igra se sastoji od dva igrača. 
Svakom je igraču dan skup od $K$ (različitih) brojeva između $1$ i $N$. 
Svaki igrač ima pristup jedino brojevima vlastitog skupa, a cilj igre 
pronaći je veličinu presjeka danih skupova. 
Igrači ne mogu komunicirati direktno, već samo koristeći zajedničku ploču 
na koju mogu postavljati žetone. 
Pravila igre su sljedeća:
\begin{itemize}[noitemsep,topsep=0pt,parsep=0pt,partopsep=0pt]
  \item 
  Ploča se sastoji od $N$ praznih polja na kojima se mogu postavljati žetoni. 

  \item 
  Igrači naizmjenično postavljaju žetone na željeno slobodno polje. 
  Jednom kada je žeton postavljen na polje, ono se smatra zauzetim i na njega se više 
  ne mogu postavljati drugi žetoni. 

  \item 
  Prvi igrač postavlja plave žetone, a drugi igrač crvene.

  \item 
  Oba igrača u svakom trenutku imaju pregled nad cijelim stanjem ploče. 

  \item 
  Kada je igrač na potezu, umjesto postavljanja žetona, on može odlučiti 
  završiti igru tako da proglasi veličinu presjeka danih skupova. 
  Ukoliko nema slobodnih polja, igrač mora završiti igru. 
\end{itemize}
Tijekom igre igrači mogu komunicirati isključivo preko ploče, no naravno, 
prije početka igre moraju se dogovoriti oko strategije. 

MALNAR je odlučio da igrače u ovoj igri treba zamijeniti 
automatiziranim sustavom umjetne inteligencije koji ima nepogrešivu strategiju te koji 
može odigrati potez odmah nakon učitavanja stanja ploče. 
Pomozite MALNAR-u te osmislite strategiju za oba igrača 
koja osigurava da barem jedan od igrača u nekom trenutku završi igru i proglasi 
točnu veličinu presjeka danih skupova. 

Da bi automatizirani sustav mogao brzo raditi poteze,  
strategija će se sastojati u tome da se za svako moguće stanje 
ploče odredi koji potez treba odigrati. 
To znači da automatizirani sustav neće imati pregled nad nizom poteza
koji je doveo igru do trenutnog stanja, 
već mora moći napraviti potez isključivo na temelju danog trenutnog stanja ploče.  

%%%%%%%%%%%%%%%%%%%%%%%%%%%%%%%%%%%%%%%%%%%%%%%%%%%%%%%%%%%%%%%%%%%%%%
% Input
\subsection*{Ulazni podaci}

U prvom je retku prirodan broj $P$ ($P=1$ ili $P=2$) koji određuje 
radi li se o prvom ili drugom igraču. 

U drugom su retku prirodni brojevi $N$ i $K$ iz teksta zadatka. 

U trećem je retku niz $a_1, a_2, \dots, a_K$ od $K$ različitih 
prirodnih brojeva ($1 \leq a_i \leq N$) 
koji predstavljaju skup brojeva dan promatranom igraču. 

U četvrtom je retku prirodan broj $T$ koji određuje broj stanja ploče za koje je 
potrebno odigrati potez. 
U testnim podacima, $T$ će uvijek biti jednak ukupnom broju mogućih stanja ploče, 
što znači da odabrana strategija mora odrediti potez za svako moguće stanje ploče. 
Međutim, samo za potrebe probnog primjera, $T$ može poprimiti i manju vrijednost. 

U svakom od sljedećih $T$ redaka je opis jednog od mogućih stanja ploče. 
Opis stanja ploče sastoji se od niza od $N$ znakova “\texttt{P}”, “\texttt{C}” ili “\texttt{.}”, 
pri čemu znak “\texttt{P}” predstavlja plavi žeton, “\texttt{C}” predstavlja crveni žeton, 
a “\texttt{.}” predstavlja prazno polje.  

%%%%%%%%%%%%%%%%%%%%%%%%%%%%%%%%%%%%%%%%%%%%%%%%%%%%%%%%%%%%%%%%%%%%%%
% Output
\subsection*{Izlazni podaci}

Za svaki od $T$ zadanih stanja ploče ispišite 
po jedan redak oblika “\texttt{+} $m$” ili “\texttt{!} $m$”, za neki cijeli broj $m$. 
\vspace{-0.15em}

Ispis oblika “\texttt{+} $m$” predstavlja postavljanje žetona 
na $m$-tu poziciju na ploču.
Da bi se ispis smatrao valjanim, mora vrijediti $1 \leq m \leq N$, 
te odabrano polje ne smije biti zauzeto. 

Ispis oblika “\texttt{!} $m$” predstavlja proglašenje da veličina presjeka danih skupova 
iznosi $m$ te završetak igre. 
Da bi se ispis smatrao valjanim, mora vrijediti $0 \leq m \leq N$. 

%%%%%%%%%%%%%%%%%%%%%%%%%%%%%%%%%%%%%%%%%%%%%%%%%%%%%%%%%%%%%%%%%%%%%%
% Scoring
\subsection*{Bodovanje}

Vaše će rješenje biti testirano u dva koraka. 
Prvo će biti pozvano na testnom podatku u kojem je $P = 1$, a nakon toga 
na testnom podatku u kojem je $P = 2$. 
Vrijednosti ulaznih podataka $N$ i $K$ bit će jednake prilikom oba pokretanja. 
Uz pretpostavku da je ispis vašeg programa valjan prilikom oba pokretanja, 
program izrađen od strane organizatora simulirat će tijek igre, prateći 
ispisani opis strategije. 
Ako navedena simulacija igre završi 
s ispravno određenom veličinom presjeka danih skupova, vaše rješenje 
smatrat će se točnim.
Vrijeme izvršavanja vašeg rješenja je zbroj vremena izvršavanja oba koraka evaluacije. 

U svim podzadacima vrijedi $2 \leq N \leq 16$ i $1 \leq K \leq N$.

{\renewcommand{\arraystretch}{1.4}
  \setlength{\tabcolsep}{6pt}
  \begin{tabular}{ccl}
   Podzadatak & Broj bodova & Ograničenja \\ \midrule
    1 & 11 & Dani skupovi sastojat će se od $K$ uzastopnih vrijednosti. \\
    2 & 7 & $N$ je paran i vrijednosti brojeva u danim skupovima su između $1$ i $\frac{N}{2}$. \\
    3 & 16 & $N \leq 4$ \\
    4 & 13 & $N = 14$ i $K = 2$ \\
    5 & 12 & Vrijednosti brojeva u danim skupovima su između $1$ i $N - 1$. \\
    6 & 41 & Nema dodatnih ograničenja.
\end{tabular}}

%%%%%%%%%%%%%%%%%%%%%%%%%%%%%%%%%%%%%%%%%%%%%%%%%%%%%%%%%%%%%%%%%%%%%%
% Examples
\subsection*{Probni primjeri}
\begin{tabularx}{\textwidth}{X'X}
\sampleinputs{text_dummy/automatizacija.dummy.in.1}{text_dummy/automatizacija.dummy.out.1} &
\sampleinputs{text_dummy/automatizacija.dummy.in.2}{text_dummy/automatizacija.dummy.out.2}
\end{tabularx}

\textbf{Pojašnjenje probnih primjera:} 

Navedeno predstavlja samo jedan primjer moguće strategije. 
Ispod je naveden odgovarajući tijek igre. 

{\renewcommand{\arraystretch}{1.4}
  \setlength{\tabcolsep}{6pt}
  \begin{tabular}{lcl}
    Stanje ploče & Potez & Napomena \\ \midrule
    \texttt{....} & \texttt{+ 1} & Prvi igrač postavlja žeton na prvo polje. \\
    \texttt{P...} & \texttt{+ 3} & Drugi igrač postavlja žeton na treće polje. \\
    \texttt{P.C.} & \texttt{+ 4} & Prvi igrač postavlja žeton na četvrto polje. \\
    \texttt{P.CP} & \texttt{+ 2} & Drugi igrač postavlja žeton na drugo polje. \\
    \texttt{PCCP} & \texttt{\frenchspacing! 1} & Prvi igrač zaustavlja igru te 
    proglašava da veličina presjeka iznosi 1. Točno! \\

\end{tabular}}

%%%%%%%%%%%%%%%%%%%%%%%%%%%%%%%%%%%%%%%%%%%%%%%%%%%%%%%%%%%%%%%%%%%%%%
% We're done
\end{statement}

%%% Local Variables:
%%% mode: latex
%%% mode: flyspell
%%% ispell-local-dictionary: "croatian"
%%% TeX-master: "../hio.tex"
%%% End:
