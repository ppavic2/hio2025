%%%%%%%%%%%%%%%%%%%%%%%%%%%%%%%%%%%%%%%%%%%%%%%%%%%%%%%%%%%%%%%%%%%%%%
% Problem statement
\begin{statement}[
  problempoints=100,
  timelimit=3 sekunde,
  memorylimit=512 MiB,
]{Lirili Larila}

Sokrat: 
\textit{Reci Platone, slažeš li se sa mnom oko ovoga: 
najjači borci su oni leteći, poput Bombardira Crocodilla ili Bombombinija Gusinija.}

Platon:
\textit{To naprosto nije tako. Kopneni borci, poput Brr Brr Patapima ili Tung Tung Tung Sahura 
postigli su svoje rezultate usprkos tome što nemaju mogućnost letenja. }

Sokrat:
\textit{Smatram da je jedini način da dođemo do istine da pustimo borce da se bore te 
da odredimo ishod na temelju toga. }

Platon: 
\textit{Bravo Sokrate, slažem se da ćemo tako doći do istine. }

Odlučujuća borba odvijat će se na povezanom grafu s $N$ čvorova i $M$ bridova. 
Lirili Larila, polu slonica polu kaktus, vlasnica je grafa pa će osigurati 
da je riječ o njezinoj najdražoj vrsti grafa: kaktus grafu. 
Za potrebe ovog zadatka, \textit{kaktus graf} definiramo kao jednostavan povezani graf 
u kojem svaki čvor pripada najviše jednom ciklusu. 

Borba se odvija na sljedeći način. 
Na početku, svi leteći borci smješteni su u određenom početnom čvoru, a svi kopneni 
borci smješteni su u nekom drugom početnom čvoru. 
Kako se borba odvija, borci šire svoj utjecaj na graf te nastoje pokoriti što više čvorova. 
U konačnici, čvor će biti pokoren od strane ili letećih ili kopnenih boraca, ovisno o tome 
je li udaljenost toga čvora bliža početnom čvoru letećih boraca ili početnom čvoru 
kopnenih boraca. 
Čvorovi koji se nalaze na jednakoj udaljenosti od početnih čvorova letećih i kopnenih boraca 
predstavljaju veliki izazov za obje skupine boraca 
pa oni ostaju nepokoreni.

Lirili Larila želi namjestiti ishod borbe. Naime, ona je već unaprijed 
odredila prirodne brojeve $A$ i $B$, koji predstavljaju broj pokorenih 
čvorova redom od strane letećih i kopnenih boraca. 
Pomozite ovoj umiljatoj kaktus-slonici da odabere početne čvorove za obje vrste boraca 
tako da na kraju borbe brojevi pokorenih čvorova odgovaraju brojevima $A$ i $B$, 
ili odredite kako takav odabir nije moguć.

%%%%%%%%%%%%%%%%%%%%%%%%%%%%%%%%%%%%%%%%%%%%%%%%%%%%%%%%%%%%%%%%%%%%%%
% Input
\subsection*{Ulazni podaci}

U prvom su retku prirodni brojevi $N, M, A$ i $B$, redom broj čvorova i broj bridova 
u danom kaktus grafu te brojevi pokorenih čvorova redom od strane letećih i kopnenih boraca.

U sljedećih $M$ redaka nalaze se parovi brojeva $a$ i $b$ ($1 \leq a, b \leq N, a \ne b$), 
redom bridovi grafa. 

Dani graf bit će kaktus graf, to jest povezani graf u kojem svaki čvor pripada najviše jednom 
ciklusu. 

%%%%%%%%%%%%%%%%%%%%%%%%%%%%%%%%%%%%%%%%%%%%%%%%%%%%%%%%%%%%%%%%%%%%%%
% Output
\subsection*{Izlazni podaci}

Ukoliko je nemoguće napraviti izbor početnih čvorova koji zadovoljava uvjete zadatka, 
u prvi i jedini redak ispišite \texttt{-1}. 

Inače, u prvi i jedini redak ispišite dva prirodna broja odvojena razmakom, koja 
predstavljaju oznake počentih čvorova letećih i kopnenih boraca. 

%%%%%%%%%%%%%%%%%%%%%%%%%%%%%%%%%%%%%%%%%%%%%%%%%%%%%%%%%%%%%%%%%%%%%%
% Scoring
\subsection*{Bodovanje}

U svim podzadacima vrijedi $2 \leq N \leq 200~000$ te $1 \leq M \leq 400~000$. 

{\renewcommand{\arraystretch}{1.4}
  \setlength{\tabcolsep}{6pt}
  \begin{tabular}{ccl}
   Podzadatak & Broj bodova & Ograničenja \\ \midrule
    1 & 6 & $N \leq 300$\\
    2 & 8 & Dani graf je stablo te $N \leq 5000$. \\
    3 & 25 & Dani graf je stablo. \\
    4 & 13 & Dani graf ima točno jedan ciklus te $N \leq 5000$. \\
    5 & 17 & Dani graf ima točno jedan ciklus te je garantirano da postoji rješenje  \\
      &    & u kojem se oba početna čvora nalaze unutar tog ciklusa.  \\
    6 & 8 & Dani graf ima točno jedan ciklus. \\
    7 & 11 & $N \leq 5000$ \\
    8 & 12 & Nema dodatnih ograničenja. \\
\end{tabular}}

%%%%%%%%%%%%%%%%%%%%%%%%%%%%%%%%%%%%%%%%%%%%%%%%%%%%%%%%%%%%%%%%%%%%%%
% Examples
\subsection*{Probni primjeri}
\begin{tabularx}{\textwidth}{X'X'X}
\sampleinputs{test/lirili_larila.dummy.in.1}{test/lirili_larila.dummy.out.1} &
\sampleinputs{test/lirili_larila.dummy.in.2}{test/lirili_larila.dummy.out.2} &
\sampleinputs{test/lirili_larila.dummy.in.3}{test/lirili_larila.dummy.out.3}
\end{tabularx}

\textbf{Pojašnjenje prvog probnog primjera:}

[slika]

\textbf{Pojašnjenje trećeg probnog primjera:}

[slika]

%%%%%%%%%%%%%%%%%%%%%%%%%%%%%%%%%%%%%%%%%%%%%%%%%%%%%%%%%%%%%%%%%%%%%%
% We're done
\end{statement}

%%% Local Variables:
%%% mode: latex
%%% mode: flyspell
%%% ispell-local-dictionary: "croatian"
%%% TeX-master: "../hio.tex"
%%% End: