%%%%%%%%%%%%%%%%%%%%%%%%%%%%%%%%%%%%%%%%%%%%%%%%%%%%%%%%%%%%%%%%%%%%%%
% Problem statement
\begin{statement}[
  problempoints=100,
  timelimit=1 second,
  memorylimit=512 MiB,
]{Korupcija}

\textit{... Corruption for all, not just for some. I offer corruption, a corrupt order, work, and growth. 
Whatever these other masters offer you, I offer double.  
I even propose an eighth case: To whom? How much? ...}

Little Mirko was fascinated by the speech of the uncle on television.  
He was convinced he understood the message: he had to corrupt the bits of his binary numbers.

Mirko considers the numbers $0, 1, \dots, 2^N-1$ (viewed as binary numbers with $N$ binary digits).  
Driven by his desire for corruption, Mirko will choose two numbers $X$ and $Y$ ($0 \leq X, Y < 2^N$) 
that differ in exactly one bit.  
He will then overwrite that bit with a “\texttt{?}” symbol in both numbers $X$ and $Y$, 
thus achieving corruption: the numbers $X$ and $Y$ can no longer be distinguished.  
Mirko will repeat this process with the remaining numbers until he obtains 
exactly $2^{N-1}$ pairs of numbers that cannot be distinguished.  
In other words, each number between $0$ and $2^N-1$ belongs to exactly one pair, and  
two numbers can form a pair if and only if they differ in exactly one bit.

For an extra challenge, Mirko decides he wants exactly $a_i$ pairs 
where the overwritten “\texttt{?}” symbol is at the $i$-th bit position, 
for each $i = 0, 1, \dots, N-1$.  
Here, bits are numbered from least significant to most significant, so the $i$-th bit 
corresponds to the value $2^i$.  
Help Mirko by choosing the pairs to satisfy the desired conditions, or determine 
that it is impossible to do so.

%%%%%%%%%%%%%%%%%%%%%%%%%%%%%%%%%%%%%%%%%%%%%%%%%%%%%%%%%%%%%%%%%%%%%%
% Input
\subsection*{Input}

The first line contains a natural number $N$ as described above.

The second line contains $N$ non-negative integers $a_i$, for $i = 0, \dots, N-1$,  
where $a_i$ represents the required number of pairs that differ at the $i$-th bit position.  
The sum of all $a_i$ is exactly $2^{N-1}$.

%%%%%%%%%%%%%%%%%%%%%%%%%%%%%%%%%%%%%%%%%%%%%%%%%%%%%%%%%%%%%%%%%%%%%%
% Output
\subsection*{Output}

If it is impossible to form pairs satisfying the required conditions, 
output a single line containing \texttt{-1}.

Otherwise, output $2^{N-1}$ lines. Each line should contain two space-separated integers $X$ and $Y$, 
representing a selected pair.  
You may output the pairs in any order.

If multiple solutions exist, output any.

%%%%%%%%%%%%%%%%%%%%%%%%%%%%%%%%%%%%%%%%%%%%%%%%%%%%%%%%%%%%%%%%%%%%%%
% Scoring
\subsection*{Scoring}

In all subtasks, it holds that $1 \leq N \leq 20$.

In every subtask, 20\% of the points are awarded for simply determining 
whether it is possible to satisfy the conditions.  
For these points, if you output anything other than \texttt{-1}, 
you may print any pairing (even if it does not fully satisfy the required condition).

{\renewcommand{\arraystretch}{1.4}
  \setlength{\tabcolsep}{6pt}
  \begin{tabular}{ccl}
   Subtask & Points & Constraints \\ \midrule
    1 & 15 & $N \leq 4$ \\
    2 & 15 & $N \geq 2$ and $a_i = 0$ for all $i > 2$ \\
    3 & 20 & $N \leq 6$ \\
    4 & 50 & No additional constraints. \\
\end{tabular}}

%%%%%%%%%%%%%%%%%%%%%%%%%%%%%%%%%%%%%%%%%%%%%%%%%%%%%%%%%%%%%%%%%%%%%%
% Sample Cases
\subsection*{Sample Cases}
\begin{tabularx}{\textwidth}{X'X'X}
\sampleinputs{test/korupcija.dummy.in.1}{test/korupcija.dummy.out.1} &
\sampleinputs{test/korupcija.dummy.in.2}{test/korupcija.dummy.out.2} &
\sampleinputs{test/korupcija.dummy.in.3}{test/korupcija.dummy.out.3}
\end{tabularx}

%%%%%%%%%%%%%%%%%%%%%%%%%%%%%%%%%%%%%%%%%%%%%%%%%%%%%%%%%%%%%%%%%%%%%%
% We're done
\end{statement}
