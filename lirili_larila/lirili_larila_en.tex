%%%%%%%%%%%%%%%%%%%%%%%%%%%%%%%%%%%%%%%%%%%%%%%%%%%%%%%%%%%%%%%%%%%%%%
% Problem statement
\begin{statement}[
  problempoints=100,
  timelimit=2 seconds,
  memorylimit=512 MiB,
]{Lirili Larila}

Socrates:  
\textit{Tell me, Plato, do you agree with me on this:  
the strongest warriors are those who can fly, like Bombardiro Crocodillo or Bombombini Gusini?}

Plato:  
\textit{That is simply not the case. Land warriors, such as Brr Brr Patapim and Tung Tung Tung Sahur, 
have achieved their success despite their inability to fly.}

Socrates:  
\textit{I believe the only way to find the truth is to let the warriors fight, and  
determine the outcome based on that.}

Plato:  
\textit{Bravo, Socrates, I agree that this is the way to reach the truth.}

The decisive battle will take place on a connected graph with $N$ vertices and $M$ edges.  
Lirili Larila, a half-elephant half-cactus creature, owns the graph and insists that it is of her favorite type: a cactus graph.  
For the purposes of this problem, a \textit{cactus graph} is defined as a simple connected graph 
in which each vertex belongs to at most one cycle.

The battle unfolds as follows:  
Initially, all flying warriors are placed at one designated starting vertex, 
and all land warriors are placed at a different designated starting vertex.  
As the battle progresses, the warriors spread their influence across the graph, attempting to conquer as many vertices as possible.  
Ultimately, a vertex is conquered either by the flying warriors or the land warriors, depending on  
whether it is closer to the starting vertex of the flying warriors or that of the land warriors.  
Vertices that are equidistant from both starting vertices  
remain unconquered, as they pose a significant challenge to both sides.

Lirili Larila wishes to control the outcome of the battle.  
She has already predetermined two natural numbers $A$ and $B$, 
representing the number of vertices to be conquered by the flying and land warriors, respectively.  
Help this lovable cactus-elephant choose starting vertices for both types of warriors 
so that, at the end of the battle, the number of conquered vertices matches the values $A$ and $B$.

Additionally, you must find such a choice for $T$ different scenarios.

%%%%%%%%%%%%%%%%%%%%%%%%%%%%%%%%%%%%%%%%%%%%%%%%%%%%%%%%%%%%%%%%%%%%%%
% Input
\subsection*{Input}

The first line contains a natural number $T$, the number of different scenarios.

Each scenario is described as follows:

The first line contains four natural numbers $N$, $M$, $A$, and $B$, representing 
the number of vertices and edges in the cactus graph, and the number of vertices 
to be conquered by the flying and land warriors, respectively.

Each of the next $M$ lines contains two integers $a$ and $b$ ($1 \leq a, b \leq N$, $a \neq b$),  
representing an edge of the graph.

The given graph is guaranteed to be a cactus graph — that is, a simple connected graph 
in which each vertex belongs to at most one cycle.

The test data will guarantee that it is always possible to find a valid choice of starting vertices.

%%%%%%%%%%%%%%%%%%%%%%%%%%%%%%%%%%%%%%%%%%%%%%%%%%%%%%%%%%%%%%%%%%%%%%
% Output
\subsection*{Output}

Print $T$ lines, one for each scenario.

In the $i$-th line, output two space-separated natural numbers, 
representing the chosen starting vertices for the flying and land warriors 
in the $i$-th scenario.  
If multiple solutions exist, you may output any.

%%%%%%%%%%%%%%%%%%%%%%%%%%%%%%%%%%%%%%%%%%%%%%%%%%%%%%%%%%%%%%%%%%%%%%
% Scoring
\subsection*{Scoring}

In all subtasks, it holds that $2 \leq N \leq 200\,000$ and $2 \leq A + B \leq N$.  
Additionally, the sum of all $N$ over all scenarios is at most $200\,000$.

The constraints listed below apply individually to each of the $T$ given scenarios.

{\renewcommand{\arraystretch}{1.4}
  \setlength{\tabcolsep}{6pt}
  \begin{tabular}{ccl}
   Subtask & Points & Constraints \\ \midrule
    1 & 6 & The sum of all $N$ is $\leq 300$. \\
    2 & 8 & The given graph is a tree and the sum of all $N$ is $\leq 5000$. \\
    3 & 25 & The given graph is a tree. \\
    4 & 13 & The given graph has exactly one cycle and the sum of all $N$ is $\leq 5000$. \\[5pt]
    5 & 17 & \makecell[l]{The given graph has exactly one cycle, and it is guaranteed \\ that a solution exists where both starting vertices are within that cycle.} \\[6pt]
    6 & 8 & The given graph has exactly one cycle. \\
    7 & 11 & The sum of all $N$ is $\leq 5000$. \\
    8 & 12 & No additional constraints. \\
\end{tabular}}

%%%%%%%%%%%%%%%%%%%%%%%%%%%%%%%%%%%%%%%%%%%%%%%%%%%%%%%%%%%%%%%%%%%%%%
% Sample Cases
\subsection*{Sample Cases}
\begin{tabularx}{\textwidth}{X'X'X}
\sampleinputs{test/lirili_larila.dummy.in.1}{test/lirili_larila.dummy.out.1} &
\sampleinputs{test/lirili_larila.dummy.in.2}{test/lirili_larila.dummy.out.2} &
\sampleinputs{test/lirili_larila.dummy.in.3}{test/lirili_larila.dummy.out.3}
\end{tabularx}

\textbf{Explanation of the First Sample Case:} \\
The flying warriors conquer vertices 4, 5, and 6, while the land warriors conquer vertex 3.  
Vertices 1 and 2 remain unconquered.

\textbf{Explanation of the Second Sample Case:} \\
The flying warriors conquer vertices 1, 2, and 4, while the land warriors conquer vertices 5 and 6.  
Vertex 3 remains unconquered.

\textbf{Explanation of the Third Sample Case:} \\
The flying warriors conquer vertices 4, 5, and 6, while the land warriors conquer vertices 1, 2, and 3.  
There are no unconquered vertices.

%%%%%%%%%%%%%%%%%%%%%%%%%%%%%%%%%%%%%%%%%%%%%%%%%%%%%%%%%%%%%%%%%%%%%%
% We're done
\end{statement}
