%%%%%%%%%%%%%%%%%%%%%%%%%%%%%%%%%%%%%%%%%%%%%%%%%%%%%%%%%%%%%%%%%%%%%%
% Problem statement
\begin{statement}[
  problempoints=100,
  timelimit=1 sekunda,
  memorylimit= 512 MiB,
]{Korupcija}

\textit{... Korupcija svima, a ne samo njima. Ja nudim korupciju, koruptivni red, rad i rast. 
Sve što vam ovi majstori ponude, ja nudim duplo. 
Predlažem i osmi padež: Kome? Koliko? ...}

Mali Mirko bio je očaran govorom stričeka s televizije. 
Bio je uvjeren kako je razumio poruku: morao je korumpirati bitove svojih binarnih brojeva. 

Mirko promatra brojeve $0, 1, \dots, 2^N-1$ (kao binarne brojeve s $N$ binarnih znamenki). 
Vođen željom za korupcijom, Mirko će izabrati dva broja $X$ i $Y$ ($0 \leq X, Y < 2^N$) 
koja se razlikuju u točno jednom bitu. Mirko će tada prebrisati 
taj bit znakom “\texttt{?}” u oba broja $X$ i $Y$, 
čime je postigao da se brojevi $X$ i $Y$ više ne mogu razlikovati. 
Mirko će ponavljati ovaj postupak s preostalim brojevima, dok na kraju ne dobije 
ukupno $2^{N-1}$ parova brojeva koji se ne mogu razlikovati. 
Dakle, svaki broj između $0$ i $2^N - 1$ član je točno jednog para i 
dva broja mogu biti u paru isključivo ako se razlikuju u točno jednom bitu (binarnoj znamenci). 

Radi većeg izazova, Mirko je odlučio da želi imati točno $a_i$ parova 
kojima znak “\texttt{?}” stoji na mjestu $i$-tog bita, za $i = 0, 1, \dots, N-1$. 
Pri tome, bitove brojimo od manje značajnih do više značajnih, pa tako $i$-ti bit 
odgovara vrijednosti $2^i$. 
Pomozite Mirku te napravite odabir parova koji zadovoljava tražene uvjete, ili 
odredite kako takav odabir ne postoji. 

%%%%%%%%%%%%%%%%%%%%%%%%%%%%%%%%%%%%%%%%%%%%%%%%%%%%%%%%%%%%%%%%%%%%%%
% Input
\subsection*{Ulazni podaci}

U prvom je retku prirodan broj $N$ iz teksta zadatka. 

U drugom je retku niz od $N$ nenegativnih cijelih brojeva $a_i$, za $i = 0, \dots, N-1$, 
pri čemu $a_i$ predstavlja traženi broj parova koji se razlikuju u $i$-tom bitu. 
Zbroj tih brojeva iznosi točno $2^{N-1}$. 

%%%%%%%%%%%%%%%%%%%%%%%%%%%%%%%%%%%%%%%%%%%%%%%%%%%%%%%%%%%%%%%%%%%%%%
% Output
\subsection*{Izlazni podaci}

Ukoliko nije moguće napraviti odabir parova koji zadovoljava uvjete zadatka, 
u jedini redak ispišite \texttt{-1}. 

Inače, ispišite $2^{N-1}$ redaka. U svaki redak ispišite dva razmakom odvojena broja $X$ i $Y$ 
koji predstavljaju odabrani par. Parove možete ispisati u bilo kojem redoslijedu.

Ukoliko postoji više rješenja, ispišite bilo koje. 

%%%%%%%%%%%%%%%%%%%%%%%%%%%%%%%%%%%%%%%%%%%%%%%%%%%%%%%%%%%%%%%%%%%%%%
% Scoring
\subsection*{Bodovanje}

U svim podzadacima vrijedi $1 \leq N \leq 20$. 

U svakom podzadatku, 20\% bodova donosi samo odlučivanje
postoji li odabir parova koji zadovoljava uvjete zadatka ili ne.  
Za te bodove potrebno je, ako niste ispisali \texttt{-1}, ispisati nekakav 
odabir parova, ali on ne mora zadovoljavati traženi uvjet.


{\renewcommand{\arraystretch}{1.4}
  \setlength{\tabcolsep}{6pt}
  \begin{tabular}{ccl}
   Podzadatak & Broj bodova & Ograničenja \\ \midrule
    1 & 15 & $N \leq 4$ \\
    2 & 15 & Vrijedi $a_i = 0$ za sve $i > 2$. \\
    3 & 20 & $N \leq 6$\\
    4 & 50 & Nema dodatnih ograničenja. \\
\end{tabular}}

%%%%%%%%%%%%%%%%%%%%%%%%%%%%%%%%%%%%%%%%%%%%%%%%%%%%%%%%%%%%%%%%%%%%%%
% Examples
\subsection*{Probni primjeri}
\begin{tabularx}{\textwidth}{X'X'X}
\sampleinputs{test/korupcija.dummy.in.1}{test/korupcija.dummy.out.1} &
\sampleinputs{test/korupcija.dummy.in.2}{test/korupcija.dummy.out.2} &
\sampleinputs{test/korupcija.dummy.in.3}{test/korupcija.dummy.out.3}
\end{tabularx}

%%%%%%%%%%%%%%%%%%%%%%%%%%%%%%%%%%%%%%%%%%%%%%%%%%%%%%%%%%%%%%%%%%%%%%
% We're done
\end{statement}

%%% Local Variables:
%%% mode: latex
%%% mode: flyspell
%%% ispell-local-dictionary: "croatian"
%%% TeX-master: "../hio.tex"
%%% End: